\documentclass[OPS,lsstdraft,authoryear,toc]{lsstdoc}
\input{meta}

% Package imports go here.

% Local commands go here.

%If you want glossaries
%\input{aglossary.tex}
%\makeglossaries

\title{Criteria to start the Legacy Survey of Space and Time}

% This can write metadata into the PDF.
% Update keywords and author information as necessary.
\hypersetup{
    pdftitle={Criteria to start the Legacy Survey of Space and Time},
    pdfauthor={First Last},
    pdfkeywords={}
}

% Optional subtitle
% \setDocSubtitle{A subtitle}

\author{%
First Last
}

\setDocRef{RTN-093}
\setDocUpstreamLocation{\url{https://github.com/lsst/rtn-093}}

\date{\vcsDate}

% Optional: name of the document's curator
% \setDocCurator{The Curator of this Document}

\setDocAbstract{%
This document concisely captures the criteria that must be satisfied to begin regular survey operations for the Legacy Survey of Space and Time (i.e. to being execution of the planned 10 year survey strategy currently documented in pstn-056). It is expected that the survey will start in late 2025 1-2 months after the beginning of the formal Operations at the completion of construction.
}

% Change history defined here.
% Order: oldest first.
% Fields: VERSION, DATE, DESCRIPTION, OWNER NAME.
% See LPM-51 for version number policy.
\setDocChangeRecord{%
  \addtohist{1}{YYYY-MM-DD}{Unreleased.}{First Last}
}


\begin{document}

% Create the title page.
\maketitle
% Frequently for a technote we do not want a title page  uncomment this to remove the title page and changelog.
% use \mkshorttitle to remove the extra pages

% ADD CONTENT HERE
% You can also use the \input command to include several content files.

\appendix
% Include all the relevant bib files.
% https://lsst-texmf.lsst.io/lsstdoc.html#bibliographies
\section{References} \label{sec:bib}
\renewcommand{\refname}{} % Suppress default Bibliography section
\bibliography{local,lsst,lsst-dm,refs_ads,refs,books}

% Make sure lsst-texmf/bin/generateAcronyms.py is in your path
\section{Acronyms} \label{sec:acronyms}
\addtocounter{table}{-1}
\begin{longtable}{p{0.145\textwidth}p{0.8\textwidth}}\hline
\textbf{Acronym} & \textbf{Description}  \\\hline

AOS & Active Optics System \\\hline
AURA & Association of Universities for Research in Astronomy \\\hline
CCR & Construction Completeness Review \\\hline
CCR1 & Construction Completeness Review 1 \\\hline
CCR2 & Construction Completeness Review 2 \\\hline
CCR3 & Construction Completeness Review 3 \\\hline
CCR4 & Construction Completeness Review 4 \\\hline
DIMM & Differential Image Motion Monitor \\\hline
DIQ & Delivered Image Quality \\\hline
DOE & Department of Energy \\\hline
DP2 & Data Preview 2 \\\hline
DR1 & Data Release 1 \\\hline
FOV & field of view \\\hline
FTE & Full-Time Equivalent \\\hline
FWHM & Full Width at Half-Maximum \\\hline
FY26 & Fiscal Year 2026 \\\hline
FoV & Field of View (also denoted FOV) \\\hline
HVAC & Heating, Ventilation, and Air Conditioning \\\hline
LPM & LSST Project Management (Document Handle) \\\hline
LSE & LSST Systems Engineering (Document Handle) \\\hline
LSR & LSST System Requirements; LSE-29 \\\hline
LSST & Legacy Survey of Space and Time (formerly Large Synoptic Survey Telescope) \\\hline
M1M3 & Primary Mirror Tertiary Mirror \\\hline
M2 & Secondary Mirror \\\hline
MREFC & Major Research Equipment and Facility Construction \\\hline
NOIRLab & NSF's National Optical-Infrared Astronomy Research Laboratory; \url{https://noirlab.edu} \\\hline
NSF & National Science Foundation \\\hline
OPS & Operations \\\hline
ORR & Operations Readiness Review \\\hline
ORR1 & Operations Readiness Review 1 \\\hline
ORR2 & Operations Readiness Review 2 \\\hline
OSS & Observatory System Specifications; LSE-30 \\\hline
PSF & Point Spread Function \\\hline
PSTN & Project Science Technical Note \\\hline
QE & quantum efficiency \\\hline
RDO & Rubin Directors Office \\\hline
RINGSS &  \\\hline
RTN & Rubin Technical Note \\\hline
SA & System and Services Acquisition \\\hline
SCOC & Survey Cadence Optimization Committee \\\hline
SLAC & SLAC National Accelerator Laboratory \\\hline
SRD & LSST Science Requirements; LPM-17 \\\hline
SV & Science Validation \\\hline
TBD & To Be Defined (Determined) \\\hline
TMA & Telescope Mount Assembly \\\hline
US & United States \\\hline
USDF & United States Data Facility \\\hline
\end{longtable}

% If you want glossary uncomment below -- comment out the two lines above
%\printglossaries





\end{document}
