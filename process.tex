\section{Process to decide when to begin the LSST}

In the following, we describe the process to reach the decision to start the LSST given an understanding of the state of the system and it's performance as outlined above. We envision three phases to be able to successfully start the LSST.  These are 1 Preparation phase; 2 Implementation phase; 3 Decision,$-$Approval$-$Launch phase. 

\subsection{Preparation Phase (now - Oct 25th)}

We are actively setting up an Early Operations Optimization Team which is distinct from the organization in the Rubin Operations steady state plan \citedsp{RDO-018}. This team is largely made up of staff from the Construction era SIT-COM (System Integration and Test $-$ Commissioning)  team who have just finished commissioning the system. This team, it's daily workflows and planning coupled with the nightly execution of observations by the Nighttime Operations Team is best placed to drive the final performance gains needed to get to LSST in this Early Operations period. For now, we envision this period to be Q1 FY26, but it could be shorter or longer. The team reports up through the AD for Rubin Summit Operations (RSO) and ten to the Directorate, 

The Rubin Directorate in Operations has started discussion with NOIRLab Communications, SLAC Communications, Rubin EPO as well as SLAC and NOIRLab leadership on public engagement and Press Releases surrounding the start of the survey (including alert generation). Our plan is to develop the press release and engagement strategy so that it is ready when we are for the survey start. 

We will set up with the Rubin team a Start of LSST Board (SLB). This board will consist of key members of the Early Operations Optimization team and is in fact a precursor to the Data Release Board \citep[DRB; see]{RDO-018}. The membership will include the ADs of the operations departments, the Head of LSST and the Deputy Head of LSST, and the former leads of integration, commissioning, and science validation from SIT-COM. This board will monitor progress on performance optimization and recommend to the Directorate when the system is ready for the start of the LSST. The SLB will be chaired by the Head of LSST.

In this preparation phase, the Early Operations Optimization team will flesh out plans for engineering observations and performance tracking with the rest of the Rubin Operations team. Nightly workflows (observations, tests, analysis, metrics) will be defined and ready for the return to on-sky observing. Rubin System Performance will be rolling out key performance indicator (KPI) tools and tracking to assist with the ultimate decision to launch the survey. These tools and KPIs will continue to be developed throughout the LSST. The criteria and process to reach a decision for the start of the survey will be fully defined in this phase including how the criteria will be measured against on-sky performance. 

Finally, as the preparation phase ends, we will communicate with our stakeholders and advisory/management bodies so that everyone knows what to expect as we emerge from the engineering shutdown and pre-survey maintenance period. We will confer with the SAC on RTN-093, and seek final approval for the criteria and process from the Rubin Management Board (RMB). The agencies will be notified of our plans. 

\subsection {Implementation Phase (Oct 25 - Launch)}

As we go back on sky, the Operations team will work with the Construction team on daily needs for high priority nighttime work as well as the punch list of items to finish construction. It is expected the punch list will require very few nights off sky. 

In the beginning, most the observing will be engineering blocks to gather data on the changes to the system made as a result of the analysis of the data taken prior to the shutdown. Periodic observations sets (Blocks) will be obtained in survey mode (known as Feature--Based Scheduler, or FBS, mode) to gauge our trajectory in the performance space (e.g. Figure~\ref{speed3}. FBS Blocks will be sufficient in length to measure effective survey speed, median DIQ (and ellipticity and focal plane uniformity). The SLB will review the periodic performance and formally vote on starting the survey. If a consensus vote is not obtained, the Early Operations Optimization team, led by the AD or RSO, will make further tests for further analysis. This cycle will be repeated until consensus is reached. 

In this period, we will communicate with stakeholders on progress. The communications group will ready our press release and other engagement elements so that we can engage quickly when the decision is made to start the survey. 

\subsection{Decision$-$Approval$-$Launch phase}
Once the SLB votes to start, they will bring this recommendation to Rubin Directorate. If the Directorate concurs with the recommendation, they will take it to the RMB for concurrence. If not, the Directorate will work with the SLB to resolve any final concerns. Once the RMB has concurred with the start decision, the agencies will be notified through the JOG and the survey will start. Assuming a successful start, the Early Operations Optimization team will be dissolved and members will take up their regular Operations roles. The SLB will continue as the DRB. 

