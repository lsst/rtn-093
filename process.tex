\section{Process to decide when to begin the LSST}

In the following, we describe the process to reach the decision to start the LSST given an understanding of the state of the system and its performance as outlined above. We envision three phases to be able to successfully start the LSST.  These are 1 Preparation phase; 2 Implementation phase; 3 Decision,$-$Approval$-$Launch phase. 

\subsection{Preparation Phase (now - Oct 25th)}

We have set up an Early Operations Optimization Team which is distinct from the organization in the Rubin Operations steady state plan \citedsp{RDO-018}. This team is largely made up of staff from the Construction era SIT-COM (System Integration and Test $-$ Commissioning)  team who have just finished commissioning the system. This team, it's daily workflows and planning coupled with the nightly execution of observations by the Nighttime Operations Team is best placed to drive the final performance gains needed to get to LSST in this Early Operations period. For now, we envision this period to be Q1 FY26, but it could be shorter or longer. The team reports up through the AD for Rubin Summit Operations (RSO) and ten to the Directorate.

We have set up with the Rubin team a Start of LSST Board (SLB). This board will consist of key members of the Early Operations Optimization team and is in fact a precursor to the Data Release Board \citep[DRB; see]{RDO-018}. The membership will include the ADs of the operations departments, the Head of LSST and the Deputy Head of LSST, and the former leads of integration, commissioning, and science validation from SIT-COM. This board will monitor progress on performance optimization and recommend to the Directorate when the system is ready for the start of the LSST. The SLB will be chaired by the Head of LSST.

\subsection {Implementation Phase (Oct 25 - Launch)}

Going back on sky, we deploy engineering observation blocks to gather data on the changes to the system made as a result of the analysis of the data taken prior to the shutdown. Periodic observations sets (blocks) will be obtained in survey mode (known as Feature--Based Scheduler, or FBS, mode) to gauge our trajectory in the performance space (e.g. Figure~\ref{speed3}. FBS Blocks will be sufficient in length to measure effective survey speed, median DIQ (and ellipticity and focal plane uniformity). The SLB will review the periodic performance and formally vote on starting the survey. If a consensus vote is not obtained, the Early Operations Optimization team, led by the AD or RSO, will make further tests for further analysis. This cycle will be repeated until consensus is reached. 

\subsection{Decision$-$Approval$-$Launch phase}
Once the SLB votes to start, they will bring this recommendation to Rubin Directorate. If the Directorate concurs with the recommendation, they will take it to the RMB for approval.

