\section{Introduction}

By the end of October 2025 the Rubin Observatory will be substantially completed having passed its Construction Completeness Review \#3 (CCR3) and Operations Readiness Review \#2 (ORR2).  This will signify the formal handover of activities from the construction project to Rubin Operations. In practice, this means the Operations team \cite[see][]{RDO-018} will assume day-to-day responsibility of the Observatory and its regular operations: running the facility on Cerro Pach\'{o}n each night, transferring data over the long haul network to the US Data Facility (USDF), processing and archiving the data, and delivering data products to the user community via realtime alerts and scheduled data releases. 

At the time of handover the construction team will have demonstrated the overall Rubin Observatory system is {\it capable} of meeting its required science driven technical performance as articulated in the Science Requirements Document \cite{LPM-17}, LSST System Requirements \cite{LSE-29} and Observatory System Specifications \cite{LSE-30} documents.  It is expected that improvements to processes, sub-system reliability and performance consistency will be necessary in advance of formally beginning the execution of the the 10--year Legacy Survey of Space and Time (LSST). The survey strategy and it's associated cadence in the 10--year period (currently allowing for modest assumed degradation in year 1) are detailed by the Survey Cadence Optimization Committee \cite[SCOC,][]{PSTN-056}. 

In this technote, we define the set of key performance criteria that the Operations team, in consultation with operations partners SLAC and NOIRLab, the post-handover Construction team, the Rubin Management Board, funding agencies, and the science advisory committee will use to guide the decision to begin the LSST. 

The formal handover from construction to operations is scheduled to occur on October 25th, 2025. The primary science program for the Rubin Observatory -- the Legacy Survey of Space and Time -- will begin after the handover to Operations and when the key ''start'' criteria have been met.  It is expected that the LSST will begin in earnest prior to the close of calendar year 2025.

Formal transition of construction staff into the Rubin Operations organization (derived jointly from NSF's NOIRLab under AURA and DOE's SLAC Rubin Operations) will be on October 1, 2025. See section~\ref{secSched} below for the current schedule.  
