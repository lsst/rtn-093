\section{Criteria to begin the LSST}

Armed with an understanding of the system performance as outlined above and the Operations team readiness, we will use a set of objective criteria to gate the start of the LSST. These criteria will be concise and easily understandable so that the community of scientists and stakeholders counting on Rubin and the survey will have confidence the Observatory is on track.

The criteria we will use that are listed below are only to start the survey. There may be key processes within the overall system, including data management and processing that require further work after handover (beyond the construction project requirements), but unless they would prevent delaying the survey start, they are not discussed or enumerated here.

The Operations team discussed an initial set of criteria with the Science Advisory Committee and community at the 2024 Rubin Community Workshop. The current criteria have been evolved since that meeting and are shown in the table below.

\begin{table}[]
\renewcommand{\arraystretch}{2}
\small
\centering
\caption{Survey Start Criteria}\label{tab:criteria}
\begin{tabular}{|p{1in}|p{4in}|p{1.0in}|}
\hline
Criterium & Description& Status \\
\hline \hline

\makecell[l]{LSSTCam\\ Maintenance} & Before the completion of SV, it is understood whether or not off TMA Camera maintenance will be needed within the first year of Operations.& TBC \\\hline  
SRD &All science requirements that can be verified with SV data are verified or expected to be verified within 3 months of completion of SV. & TBC \\\hline
IQ& Image quality contribution of system is better than 0.7’’. & \makecell[l]{TBC, \\preliminary \\result \\ComCam <0.4$''$}\\\hline
Speed & Survey speed as measured on SV is > 0.5. & TBC, ComCam promising\\\hline
\makecell[l]{Improve\\ performance} & Both IQ and speed are understood with clear paths to improvement (as needed) that can be done in parallel with on going observing.& TBC \\\hline
Dome & The dome environment is under control at night. & TBC\\\hline
Cadence& The first six months of survey schedule includes details of the system as currently performing.& in progress\\\hline
Fit & Team is rested and fit. & TBC\\\hline
Early Science&DP2 observations are completed as planned \citep{RTN-011}.& TBC \\

\hline
\end{tabular}
\end{table}

These criteria are not comprehensive with respect to LSST success. They are intended only to guide the confident commencement of the survey. We emphasize that the initial boundary condition is a successful completion of the construction project. This means the construction completeness reviews have been successfully completed and NSF and DOE have accepted the system as the one they intended to build.

With this in mind, the most important criterion is that the system can efficiently take data. Figure~\ref{speed} shows where efficiency will be key in making progress on the survey. Beginning the scheduled observations no sooner that the system can run at 50$\%$ effective speed and with image quality (system contribution) below 0.7$''$ is acceptable as long as a clear path to improved performance exists. We assume the path to improvement will arrive at nominal effectiveness (1.0, 0.35$''$) within 6 months of the start of survey. Indeed, the ComCam campaign in Q1 FY24 showed the system contribution to image quality is likely as good 0.4$''$ already without full control over the in$-$dome environment \citep[See][ section 2.2]{SITCOMTN-149}.

The choice of when to begin the survey based on the effective survey speed (\ref{speed}) is subjective. We choose 50$\%$ as the point where accumulating survey data while working on performance improvement is most productive for the use of nighttime hours. This is with the proviso that clear paths to performance improvement are well understood.

In order that the data taken are of the highest quality possible in this context, the dome environment should be controllable to the extent dome seeing is significantly impacting the delivered IQ. The survey schedule should be updated with the current performance for the first six month period.

Finally, the community should have a significant data set to analyze while year one data are being obtained. So, the DP2 data taking should be complete and moving toward release. 

