\section{Criteria to begin the LSST}

The criteria to start LSST based on the needed performance described above are presented here in Table~\ref{tab:criteria}.

\begin{table}[]
\renewcommand{\arraystretch}{2}
\small
\centering
\caption{Survey Start Criteria}\label{tab:criteria}
\begin{longtable}{|p{0.25in}|p{1in}|p{4in}|p{1.0in}|}
\hline
Item & Criterion & Description& Status \\
\hline \hline

1 & sDIQ & The ``System'' contribution to the measured Delivered Image Quality is better than or equal to 0.45$\arcsec$ & \makecell[l]{\\Currently 0.6$\arcsec$ \\system \\contribution\\}\\\hline

2 & sDIQ Uniformity& The ``System'' contribution to the measured Delivered Image Quality can vary over the field of view such that 10$\%$ or less of the FOV has a system contribution of up to 0.52$\arcsec$ (see LSST system specifications LSR-REQ-0008.  & \makecell[l]{\\Currently\\ not meeting \\for typical\\images\\}\\\hline

3 & Ellipticity & Ellipticity for a single image will typically be as specified in the LSST system requirements as LSR-REQ-0092. This is $\leq$ 0.04 with 5$\%$ or fewer outliers beyond 0.07. See below; we will gate the survey start with respect to the residuals between the measured ellipticity and the calculated value from the WFS.& \makecell[l]{\\ not \\meeting\\}\\\hline

4 & Normalized \'{E}tendue (\it{eF}) & Survey Speed  is $>$ 0.7. & Currently 0.68 \\\hline

5 & Calibration & All necessary calibration data products are available at the time any LSST data are obtained or can be obtained after the fact without invalidating the observed data for inclusion in the LSST. & {Check status at CCR3} \\\hline

6 & Dome & The dome environment is not limiting typical performance. &\makecell[l]{\\Not controlled \\until after \\Handover\\}\\\hline

7 & LHN ready & The Long Haul Network will be working reliably and not be a limiting factor in Alert Production & \textcolor{dartmouthgreen}{ {LHN is ready and not limiting Alert Production}}\\\hline

8 & DM ready& Data are routinely passed from the summit and ingested into storage at the USDF. Nightly processing including calibrations are routinely running, nightly alert production is running. & \textcolor{dartmouthgreen}{These elements are all met.} \\\hline

9 & Survey Strategy Ready & The initial LSST strategy is available & \textcolor{dartmouthgreen}{This is ready (v 5.1 has been executed at night)} \\\hline

\hline
\end{longtable}
\end{table} 

\subsection{Delivered Image Quality (DIQ)}

We will start the LSST when the sustained performance of the system contribution to DIQ is $\le$ 0.45$\arcsec$.  This criteria is the combination of both the telescope $+$ LSSTCam optics and the degradations caused from any thermal non-equilibrium. The system contribution for purposes of this calculation is obtained by subtracting quadrature the estimate of the atmospheric contribution obtained from the wave front sensor donut analysis. This approximation thus excludes the dome seeing. 

Equally important for the scientific value of the images is that the typical LSST images meet the requirements for uniformity across the field of view and ellipticity for bright isolated stars. These are given in Table~\ref{tab:criteria}. For uniformity to start the survey, we identify the value of the allowed contribution in the worst quartile of the overall seeing profile given in the SRD. This is 0.52$\arcsec$. 

We will start the LSST when the residuals for ellipticity from the system component are met or well understood. This is because a calculation of the ellipticity (from the WFS) can result in residuals from the measured ellipticity which would allow us to detect the astrophysical weak lensing signal. This is described in the \citep[Section 3.3.3.3]{LPM-17}, "This specification does not by itself address weak lensing systematics, 
because there are schemes for removing the influence of an anisotropic PSF on the observed shapes of galaxies. 
However, it is known that these schemes leave smaller residuals if initially given isotropic PSFs to begin with ..." 

\subsection{Dome Environmental Control}
The dome will be the last major subsystem to be completed. Indeed it will not be done until mid 2026.  However, the total contribution to sDIQ from the dome environment is modest. Including all sources of turbulence generated by the facility, the budget (i.e residual contribution after control) to contribute to the DIQ is only 0.09$\arcsec$. We will start the LSST before the Dome work is complete.

\subsection{Normalized \'{E}tendue}
Much of the effective speed (Normalized  \'{E}tendue) is already demonstrated to be sufficient to start the LSST, The field of view factor, fA, is excellent and stable. The sensitivity factor, fS, is also very good. This factor has the potential to help overall performance because of its dependence on delivered image quality. Getting the system contribution to the DIQ from 0.6 to 0.4 (coupled with site free air atmospheric seeing of 0.7$\arcsec$) would raise fS from 0.94 to 1.3. Apart from this, all the optics are delivered as is the focal plane. Slew and settle performance are well understood and adequate. Modest gains on fO are forecast for the rest of SV and early operations. We expect the current value to go from 0.97 to 1.05. 

This leaves the System Availability, Current performance of 0.75 needs to be improved but will not stop the survey start.

Using the combined SRD minimum specifications the lowest Normalized \'{E}tendue allowed is 0.7. This level is nearly in hand; see Table~\ref{tab:factors}. 

There remains the question of how reliably (often) we keep the performance in the green SRD box. Clearly some images will fall outside the box for any number of reasons, even in steady state operations. For purposes of starting the LSST. we will adopt a criteria that we can begin if nightly median performance of contiguous validation runs remain in the target area for two continuous weeks. 

\begin{figure}[t]
\centering
\includegraphics[width=0.85\linewidth]{speed3.png}
\caption{Image quality versus effective survey speed or Normalized \'{E}tendue. The large green rectangle represents the region in this space within which we can confidently start the LSST while working to maintain and improve performance. We expect to enter this box near the middle top where the shading is lighter.}
\label{speed3}
\end{figure}

\newpage
